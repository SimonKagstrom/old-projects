\documentclass[a4wide]{article}
\usepackage[dvips]{graphicx}
\usepackage[T1]{fontenc}
\pagestyle{empty}

\begin{document}
\section{Introduction}

% Event-based API
% Save space
%  No error checks (almost)
%  Simple structures, i.e. lines are also of equal length
%  16 objects, only one pos possible.
% SDL development library
% Uses library


% Describe different fields (in a screenshot)
% Terminology, user - player etc

\section{Game implementation}

\subsection{Examples}

\subsubsection{Looking at the scene in general}
The goal here is to look at the current scene in general. Looking can of course
also be done at special parts of the scene, i.e. event-points, and
event-handling is described later. Looking in general is easily done - simply
check if \verb&action& is look and then display a message depending on which
scene the user is on. However, this will require fairly much code, so a more
general solution can be used instead.

The idea we use is that the scenes are numbered from 0 to width*height-1 (of the
map. If we then place the scene-descriptions adajcent to each other in the
message file, we can handle all general looks in one place. The code is shown in
figure \ref{fig:look_scene}.

The event we use is the PLAYER\_STOP event. This is raised each time the player
stops walking. In the event-handler, we check for the PLAYER\_STOP event. We
thereafter call the specific handler for these events, i.e.
\verb&handle_player_stops_events()&.  In this function, we check if the action
is look, and if so displays a message. The scene-describing messages are always
2 lines long here, and start at SCENE\_MESSAGES\_START (which is chosen so that
we have 16 object-descriptions before.

\begin{figure}
  \scriptsize
\begin{verbatim}
#define SCENE_MESSAGES_START 17

void advg_handle_event(advg_game_t *p_game, advg_event_t event)
{
  ...
  if (ADVG_EVENT_IS(event, ADVG_EVENT_PLAYER_STOP))
    handle_player_stops_events(p_game);
}

static void handle_player_stops_events(advg_game_t *p_game)
{
  switch(p_game->action)
    {
    case ADVG_ACTION_LOOK:
      advg_display_message(p_game, p_game->scene*2+SCENE_MESSAGES_START, 2);
      break;
    ...
    }
}
\end{verbatim}
  \label{fig:look_scene}
  \caption{Code to look at the scene}
\end{figure}


\subsubsection{Use an object on the player}
The goal here is to react when an object is used on the player, for instance if
the player drinks a potion. This would be represented by a press on \emph{use},
followed by a press on the object in the object-field and thereafter on the
player.

The code to use an object on the player can be seen in figure
\ref{fig:obj_on_player}. In the event-handler, it is first checked if the event
is a player-event, i.e.  if there was a press on the player. If this is the
case, \verb&handle_player_events()& is invoked. Note that the event-handler
becomes cleaner if the actual event-handling is done in functions outside it.

In the \verb&handle_player_events()& function, we first check if the current
action is \emph{use}. If \emph{use} was pressed, we continue by checking the
object number. If the user did not press an object after use, for instance
instead pressing the player directly or if the player did not have the object,
\verb&obj_use& will be 0. Otherwise, \verb&obj_use& will have a value between 1
and 16, depending on which object was pressed.

For the sequence of \emph{use}, first object, player was pressed, the
implementation should therefore be placed below \verb&case 1:& in figure
\ref{fig:obj_on_player}.

\begin{figure}
  \scriptsize
\begin{verbatim}
void advg_handle_event(advg_game_t *p_game, advg_event_t event)
{
  ...
  if (ADVG_EVENT_IS(event, ADVG_EVENT_PLAYER))
    handle_player_events(p_game); /* Player-event */
}

static void handle_player_events(advg_game_t *p_game)
{
  if (p_game->action == ADVG_ACTION_USE)
    {
      switch (p_game->obj_use)
        {
        case 0: /* No object pressed */
          break;
        case 1: /* First object */
        ...
        }
    }
  else
    ...
}
\end{verbatim}
  \normalsize
  \label{fig:obj_on_player}
  \caption{Code to use an object on the player}
\end{figure}

\end{document}